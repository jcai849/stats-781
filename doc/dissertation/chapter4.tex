\chapter{Real Applications} %reasonably short - similar to chapter 3
With the more realistic model developed in the previous chapter, we were ready to apply our Nested Sampling code to real radial velocity datasets.


\section{HD 37124}
The first star we investigated was HD 37124, a yellow dwarf star approximately 110 light years away in the Taurus constellation. It is currently believed to have three exoplanets in orbit around it with periods of 154, 886, and 1862 days. We obtained the radial velocity data from \cite{2011ApJ...730...93W}.


We ran our code with 50 repetitions of Nested Sampling, each with 100 iterations and $n = 100,000$ MCMC steps per iteration. The posterior distribution for the log periods is shown in Figure 4.1. Due to the arbitrary labelling of the planets, the posterior distributions for $T_1, T_2,$ and $T_3$ are the same. Therefore, we combined all the periods and plotted just a single posterior distribution which shows the periods of the three planets. As seen in Table 4.1 below, we mostly agree with the literature values except our $T_2$ has a smaller value whilst our $T_3$ is slightly higher. Our uncertainties are larger than the literature values so we have wider credible intervals. One possible explanation is that we used the t-distribution and the posterior distribution of $\nu$ wanted to be low, indicating that the data is heavy-tailed. Whilst our velocities for Planet 1 and Planet 2 are consistent with the literature values, we disagree on velocity for Planet 3. We believe that Planet 3 may have a more elliptical orbit than previously thought, this could also be influenced by the difference in the period for Planet 3.


\noindent
\begin{minipage}{\linewidth}% to keep image and caption on one page
\centering
\includegraphics[keepaspectratio=true,width=0.7\linewidth]{./logperiodhisthd37124post}
\captionof{figure}{\textit{Graph of the posterior distribution for the log period of all possible planets for HD 37124, where the red lines are the commonly believed values according to \cite{2011ApJ...730...93W}. We defined Planet 1 to be the one with a log period around 5, Planet 2 as the planet with a log period around 6.7, and Planet 3 as the one around 7.6.}}\label{fig:LpPost}
\end{minipage}


\begin{table}[hbtp]
\centering
\begin{tabular}{|l|r|r|c|c|}
\hline
Parameter & Mean & 68\% CI & Exponential CI & Literature Values \\ \hline
Log Period 1 & 5.0401 & 5.0380 -- 5.0419 & 154.16 -- 154.76 & 154.29 -- 154.47\\
Log Period 2 & 6.7555 & 6.7310 -- 6.7736 & 838.0 -- 874.4 & 880.4 -- 890.6\\
Log Period 3 & 7.6455 & 7.5688 -- 7.7348 & 1937 -- 2287 & 1824 -- 1900\\
Velocity 1 & 0.960 & 0.936 -- 0.985 &  NA & 0.958 -- 0.987\\
Velocity 2 & 0.944 & 0.903 -- 0.988 &  NA & 0.906 -- 0.964\\
Velocity 3 & 0.667 & 0.528 -- 0.776 &  NA & 0.84 -- 0.99\\
\hline
\end{tabular}
\caption[Table 2]{\textit{Table of the results for our model with star HD 37124 compared to the literature values. Period is measured in days.}}
\end{table}

\noindent
\begin{minipage}{\linewidth}% to keep image and caption on one page
\centering
\includegraphics[keepaspectratio=true,width=0.8\linewidth]{./mugraphHD}
\captionof{figure}{\textit{Graph of the observed radial velocity data for star HD 37124 (in black), with three proposed solutions.}}\label{fig:muHD}
\end{minipage}

Figure 4.2 shows the radial velocity data for HD 37124, with three models $y(t)$ overplotted. The green and blue curves show the oscillations due to the three planets. The fast oscillation corresponds to the planet with a period of roughly 154 days, when ignoring the movement from this planet we can see that the remaining signal is too complicated to be explained by only one other planet. The red curve may seem unreasonable, but there is nothing in the data which forbids it. The gaps in the dataset result in greater uncertainty about what happens in these sections, which is how the red curve can act more unexpectedly yet still fit the observed data.  While this unexpected behaviour is possible it is not probable and most of the proposed solutions resemble the green and blue curves. Where the gaps are longer, we see more discrepancies between the red curve and the green and blue curves.

\section{$\nu$ Ophiuchi}
$\nu$ Ophiuchi, also known as $\nu$ Oph or HD 163917, is a K giant star about 151 light years away in the constellation Ophiuchus. It is currently believed to have two very large exoplanets (also known as brown dwarves or failed stars) \cite{2012PASJ...64..135S}.

For this star system we ran our code with 25 Nested Sampling repetitions, each with 100 iterations and $n = 100,000$ MCMC steps per iteration. The posterior distribution for the log period is plotted in Figure 4.3. We dealt with the arbitrary labelling the same way with this system as with the HD 37124 star. As seen in Table 4.2 the periods for the two planets are consistent as is our value for $v_1$. However, we think that $v_2$ is slightly closer to 1, indicating a slightly more circular orbit than currently believed. We can also see a small third peak in Figure 4.3, indicating a slim possibility of a third planet with a period of $\sim$ 55 days.

 
\noindent
\begin{minipage}{\linewidth}% to keep image and caption on one page
\centering
\includegraphics[keepaspectratio=true,width=0.7\linewidth]{./logperiodnuoph}
\captionof{figure}{\textit{Graph of the posterior distribution for the log period of all possible planets for HD 37124, where the red lines are the commonly believed values \cite{2012PASJ...64..135S} \cite{2014arXiv1401.6128H}}}\label{fig:LpPostnu}
\end{minipage}


 
\noindent
\begin{minipage}{\linewidth}% to keep image and caption on one page
\centering
\includegraphics[keepaspectratio=true,width=0.7\linewidth]{./lperiodnuoph10k}
\captionof{figure}{\textit{Graph of the posterior distribution for the log period of all possible planets for HD 37124 based on Nested Sampling runs with only 10,000 MCMC steps. The red lines are the commonly believed values \cite{2012PASJ...64..135S} \cite{2014arXiv1401.6128H}}}\label{fig:LpPostnushort}
\end{minipage}



\begin{table}[hbtp]
\centering
\begin{tabular}{|l|r|r|c|c|}
\hline
Parameter & Mean & 68\% CI & Exponential CI & Literature Values \\ \hline
Log Period 1 & 6.2736 & 6.2728 -- 6.2744 & 530.0 -- 530.8 & 529.7 -- 530.1\\
Log Period 2 & 8.0656 & 8.0608 -- 8.0703 & 3168 -- 3198 & 3176 -- 3246\\
Velocity 1 & 0.9376 & 0.9331 -- 0.9418 &  NA & 0.9304 -- 0.9353\\
Velocity 2 & 0.9177 & 0.9086 -- 0.9271 &  NA & 0.8905 -- 0.9039\\
\hline
\end{tabular}
\caption[Table 2]{\textit{Table of the results for our model with star HD 163917 ($\nu$ oph) compared to the literature values \cite{2014arXiv1401.6128H}. Period is measured in days.}}
\end{table}

\noindent
\begin{minipage}{\linewidth}% to keep image and caption on one page
\centering
\includegraphics[keepaspectratio=true,width=0.8\linewidth]{./mugraphnuoph}
\captionof{figure}{\textit{Graph of the observed radial velocity data for star $\nu$ Ophiuchi (also called HD 163917) \cite{2012PASJ...64..135S}, with three proposed solutions.}}\label{fig:munuoph}
\end{minipage}

Figure 4.5 shows the radial velocity data for $\nu$ oph, with three models of $y(t)$ overplotted. The red and blue curves are so similar as to be nearly indistinguishable, whilst the green curve has some small signal relating to the potential third planet. When we ran the Nested Sampling code with only $n = 10,000$ MCMC steps (see Figure 4.4) we didn't see the third peak. Running the code for $n = 100,000$ MCMC steps meant the third peak appeared. Since the high $n$ run should be more accurate we might expect the third peak to become bigger if we increased $n$ further. 

The most common way to infer the number of planets, i.e. in this case whether it is two or three planets, is to compare the marginal likelihoods for the two models, however this is not always necessary because a three planet model can effectively act as a two planet model if one of the planets has a very low amplitude. When the amplitude is very low all the other properties of the planet will be very uncertain. We calculate the posterior probability that a third planet exists in this data with a log period between 3 and 5 to be roughly 58\%. Whilst \cite{2014arXiv1401.6128H} did see the third peak pop up they calculated that the three planet model had a lower marginal likelihood than a two planet model so was less plausible.



