% Created 2019-04-10 Wed 03:55
% Intended LaTeX compiler: pdflatex
\documentclass[11pt]{article}
\usepackage[utf8]{inputenc}
\usepackage[T1]{fontenc}
\usepackage{graphicx}
\usepackage{grffile}
\usepackage{longtable}
\usepackage{wrapfig}
\usepackage{rotating}
\usepackage[normalem]{ulem}
\usepackage{amsmath}
\usepackage{textcomp}
\usepackage{amssymb}
\usepackage{capt-of}
\usepackage{hyperref}
\date{\today}
\title{}
\hypersetup{
 pdfauthor={},
 pdftitle={},
 pdfkeywords={},
 pdfsubject={},
 pdfcreator={Emacs 26.1 (Org mode 9.1.9)}, 
 pdflang={English}}
\begin{document}

\tableofcontents

\section{2019-03-06}
\label{sec:orgde47e23}
\subsection{Preparation}
\label{sec:org7bba46c}
\textbf{Question}: How should this dissertation relate to Cassidy's
\href{../reading/CassidyStuff/CassidyStuff/Our\%20text\%20analytics\%20project.docx}{project
outline}?

Thoughts on data structures:

\begin{itemize}
\item dataframe seems ok for tidy text style data
\item nested dataframe may be best for survey response data
\item look at other styles (incl. non-tidy), maybe trees, as per parse trees
etc
\item Document-term matrix for tf-idf (PCA would be interesting here due to
massive dimension)
\end{itemize}

\textbf{Reading}: \href{https://www.tidytextmining.com}{Text Mining with R}:
\href{../notes/text\_mining\_with\_r.org}{notes}

\begin{itemize}
\item[{$\boxtimes$}] Read Text Mining with R
\item[{$\boxtimes$}] Assess Twitter api
\item[{$\boxtimes$}] Play with iNZight/lite, Cassidy's Project
\item[{$\boxtimes$}] Consider UI
\item[{$\boxtimes$}] Consider Survey Responses
\item[{$\boxtimes$}] Draft UI Depictions
\end{itemize}

\subsection{Minutes / Summary}
\label{sec:org2c30583}
\begin{quote}
UI is something that will have to be organically developed as we go.
\end{quote}

Meeting was mainly demonstration from Jason to Chris about the
preparation done, as well as Chris demonstrating to Jason the current
iNZight UX. Research so far is good, but too broad to demonstrate in the
confines of a weekly meeting - Chris suggested a \hyperref[sec:org969fe61]{solution};
to set notes in a neat summary format and publish to a github repository

\subsection{Actions}
\label{sec:org969fe61}
\begin{itemize}
\item[{$\boxtimes$}] Set notes in neat summary form (organise file structure to match)
\item[{$\boxtimes$}] Push to a private github repository
\item[{$\boxtimes$}] Give access to Chris
\item[{$\square$}] \sout{Create twitter developer account}
\item[{$\square$}] \sout{Get twitter api access token}
\end{itemize}

\section{2019-03-13}
\label{sec:org8edee7b}
\subsection{Minutes / Summary}
\label{sec:org61a6b2d}
This meeting took place while seeing what packages exist already to
complete various tasks. We looked though the various packages I have
found. Determined that we needed to reel in and pick specific features
that we want, according to the heuristic; if it isn't obvious
immediately, get rid of it. The primary question we seek to answer with
this text analytics program is, what are people talking about, and how
do they feel about it. With this in mind, analysis for e.g. writing
styles are not to be considered (who wrote shakespeare etc.).


\subsection{Actions}
\label{sec:org28cf0d8}
\begin{itemize}
\item[{$\square$}] \sout{interface (w. respect to above)}
\item[{$\square$}] \sout{lit review of twitter analysis (esp. discourse)}
\item[{$\boxtimes$}] Determine what text analysis really is at base, and how it relates
to the primary question
\item[{$\boxtimes$}] find if subtitles can be attained easily enough
\item[{$\boxtimes$}] formal critique of Cassidy's project, integration w/ iNZight
\item[{$\boxtimes$}] scope reduction
\item[{$\boxtimes$}] See similar papers in stats printing room
\end{itemize}

\section{2019-03-20}
\label{sec:org5b8770c}
\subsection{Summary}
\label{sec:org0f841a7}
Close assessment of textrank package and background algorithm - on the
right track for scope. Chris noted to check for survey response data
etc. from those involved with it.

\subsection{Actions}
\label{sec:orgf4f23dd}

\begin{itemize}
\item[{$\boxtimes$}] Create an example corpus for testing
\item[{$\boxtimes$}] Assess textrank performance on the example corpus
\end{itemize}

\section{2019-03-28}
\label{sec:org45d09cf}
\subsection{Summary}
\label{sec:org40e4774}
Presentation of further package discoveries, for development purposes,
as well as end-user. Demonstrated the benefits and drawbacks of using
textRank, discussed the possibility of lexRank. Presented draft of
feature space - including the usefulness of categorisation of text
analytics as within- and between- texts. Much discussion dedicated to
summarisation.

\subsection{Actions}
\label{sec:orgda6998a}
\begin{itemize}
\item[{$\boxtimes$}] Development of a test-corpus; Chris has emailed some free-form
patient responses
\item[{$\boxtimes$}] Further testing of textRank (try tidy-style) on corpus
\item[{$\boxtimes$}] Testing lexRank on on test corpus
\item[{$\boxtimes$}] Formalise feature-space as list
\end{itemize}

\section{2019-04-03}
\label{sec:org3f8b161}
\subsection{Summary}
\label{sec:org132af34}
Demonstration of summarisation systems, confirmation of use of lexRank,
discussion of case of free-form response data - much interest in these
capabilities

\subsection{Actions}
\label{sec:orgd1ccbad}
\begin{itemize}
\item[{$\square$}] Summarisation of Free responses using the two example datasets;
indiv. response as token?
\item[{$\square$}] Testing of standard text analysis tools on free-response data - see
what is useful
\item[{$\square$}] Test grouping variables such as survey group in the analysis
\item[{$\square$}] LexRank for keywords
\end{itemize}

\section{2019-04-03}
\label{sec:orge83fc28}
\subsection{Summary}
\label{sec:org06b65a9}

\subsection{Actions}
\label{sec:org5add0f3}

\subsection{{\bfseries\sffamily TODO} }
\label{sec:orgae31cc8}
\end{document}