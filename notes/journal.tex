% Created 2019-04-10 Wed 02:52
% Intended LaTeX compiler: pdflatex
\documentclass[11pt]{article}
\usepackage[latin1]{inputenc}
\usepackage[T1]{fontenc}
\usepackage{graphicx}
\usepackage{grffile}
\usepackage{longtable}
\usepackage{wrapfig}
\usepackage{rotating}
\usepackage[normalem]{ulem}
\usepackage{amsmath}
\usepackage{textcomp}
\usepackage{amssymb}
\usepackage{capt-of}
\usepackage{hyperref}
\date{\today}
\title{}
\hypersetup{
 pdfauthor={},
 pdftitle={},
 pdfkeywords={},
 pdfsubject={},
 pdfcreator={Emacs 26.1 (Org mode 9.1.9)}, 
 pdflang={English}}
\begin{document}

\tableofcontents

\section{\href{./TODO.md\#2019-03-06}{TODO}}
\label{sec:org6f229bb}

\section{Meetings}
\label{sec:org195fba5}
\begin{itemize}
\item \href{./meeting\_notes.md\#2019-03-06}{2019-03-06}
\item \href{./meeting\_notes.md\#2019-03-13}{2019-03-13}
\item \href{./meeting\_notes.md\#2019-03-20}{2019-03-20}
\item \href{./meeting\_notes.md\#2019-03-28}{2019-03-28}
\item \href{./meeting\_notes.md\#2019-04-03}{2019-04-03}
\end{itemize}

\section{Initial Survey}
\label{sec:org4476a5d}
\subsection{Prescribed Reading}
\label{sec:org09aad24}
\begin{itemize}
\item \href{https://www.tidytextmining.com}{Text Mining with R}: \href{./text\_mining\_with\_r.org}{notes}
\item \href{https://m-clark.github.io/text-analysis-with-R/}{Text Analysis with R}
\end{itemize}

\subsection{Consideration of Application Features}
\label{sec:orgd9675e5}

\textbf{UX/UI} must be intuitive, modelling the inherently non-linear workflow,
while fitting in with the iNZight / lite environment. I wrote some notes
on UX/UI \href{./ux\_ui.org}{here}.

\textbf{Text Classification} seems to be a hot topic in sentiment analysis, but
the question remains of whether it is within our scope (I suspect not).
If it were, \href{https://cran.r-project.org/web/packages/openNLP/}{openNLP}, along with some pre-made \href{https://datacube.wu.ac.at/src/contrib/}{models}, would likely serve 
this topic well. An interesting example of text classification in the
Dewey Decimal System is given \href{http://creatingdata.us/models/SRP-classifiers}{here}.

\textbf{\href{../reading/Thurner2015\%20-\%20Understanding\%20Zipfs\%20Law\%20of\%20Word\%20Frequencies\%20through\%20Sample\%20Space\%20Collapse\%20in\%20Sentence\%20Formation.pdf}{Zipf's Law}} is referred to regularly in text analysis. Should we demonstrate
this directly, as part of some other analysis, or not at all?

The \textbf{form of analysis} will vary enormously with different forms of
text. There are some things constant for all forms of text, but a good
deal is very specific. For example, the information of interest will
differ between novels, discourse (interviews, plays, scripts), twitter
posts, survey response data, or others. It may be worthwhile to either
focus solely on one, or give the option to specify the type of text.

\textbf{Data structures} will change based on the text type, and the packages
used. With a tidy dataframe as our base structure, it is easy enough to
convert to specific objects required by various packages, harder to
convert back.

There is a natural link between \textbf{text analysis and Linguistics}, and a
significant amount of the terminology in the field reflects that. Our
application requires far more market share than one that a mention in a
third year linguistics paper provides, and so linguistics is not going
to be our primary focus. Regardless, many forms of analysis require some
linguistic theory, such as those dependent on Part of Speech tagging, so
it is still worthwhile to keep in mind

\subsection{Twitter}
\label{sec:org8444a95}
Twitter data looks particularly interesting, as it is a constantly
updating, rich source of information. I wrote up some notes on text
mining twitter \href{./text\_mining\_twitter.org}{here}. It would be
particularly interesting to view twitter data in the context of
discourse analysis.

\subsection{Subtitles}
\label{sec:orgf916aac}
Subtitles are a unique form of text that would be very interesting to
analyse. Subtitles for films and TV Series can be obtained easily from
the site \href{https://www.opensubtitles.org/en/search/subs}{opensubtitles}, though
obtaining subtitles programatically may be more difficult. It clearly is
possible, as VLC has an inbuilt feature, as does \href{https://github.com/zerratar/SubSync}{subsync}, which is written in
C\#, so would require a port to R (probably not worth it for us at this
point). Subtitles usually come as .srt files, and once the file is
obtained, it's easy enough to import and work with it in R with the
package \href{https://github.com/fkeck/subtools}{subtools}.

\subsection{R Packages}
\label{sec:org4c32989}
\href{https://quanteda.io/articles/pkgdown/comparison.html}{Here} is a useful comparison between the major text mining packages. CRAN also has
a \href{https://cran.r-project.org/web/views/NaturalLanguageProcessing.html}{task view} specifically for Natural Language Processing, offering many
packages relevant to this project. Interestingly, they are split by
linguistic category; Syntax, Semantics, and Pragmatics. The further from
syntax the package is, the far more interesting it intuitively appears
(eg. word count vs sentiment analysis). Some packages of interest
include:

\begin{description}
\item[{\href{https://github.com/juliasilge/tidytext}{tidytext}}] is a text-mining
package using tidy principles, providing excellent interactivity with
the tidyverse, as documented in the book
\href{https://www.tidytextmining.com}{Text Mining with R}
\item[{\href{http://tm.r-forge.r-project.org/}{tm}}] is a text-mining framework
that was the go-to for text mining in R, but appears to have been made
redundant by tidytext and quanteda of late
\item[{\href{https://quanteda.io/}{quanteda}}] sits alone next to qdap in the
Pragmatics section of the NLP task view, and offers a similar
capability to tidytext, though from a more object-oriented paradigm,
revolving around \emph{corpus} objects. It also has extensions such as
offering readability scores, something that may be worth implementing.
\item[{\href{https://trinker.github.io/qdap/vignettes/qdap\_vignette.html}{qdap}}] is a "quantitative discourse analysis package", an extremely rich set
of tools for the analysis of discourse in text, such as may arise from
plays, scripts, interviews etc. Includes output on length of discourse
for agents, turn-taking, and sentiment within passages of speech. This
looks to me like the most insight that could be gained from a text.
\item[{\href{https://github.com/trinker/sentimentr}{sentimentr}}] is a rich
sentiment analysis and tokenising package, with features including
dealing with negation, amplification, etc. in multi-sentence level
analysis. An interesting feature is the ability to output text with
sentences highlighted according to their inferred sentiment
\item[{\href{https://rstudio.github.io/dygraphs/}{dygraphs}}] is a time-series
visualisation package capable of outputting very clear interactive
time-series graphics, useful for any time-series in the text analysis
module
\item[{\href{https://github.com/thomasp85/gganimate}{gganimate}}] produces  animations on top of the \href{https://github.com/tidyverse/ggplot2}{ggplot} package, offering
powerful insights. \href{https://www.r-bloggers.com/investigating-words-distribution-with-r-zipfs-law-2/}{Here} is an example demonstrating Zipf's Law
\item[{\href{https://github.com/bnosac/textrank}{textrank}}] has the unique idea
of extracting keywords automatically from a text using the pagerank
algorithm (pagerank studied in depth in STATS 320) - my exploration of
the package is documented \href{./textrank\_exploration.Rmd}{here}
\item Packages for obtaining text:

\begin{description}
\item[{\href{https://cran.r-project.org/web/packages/gutenbergr/index.html}{gutenbergr}}] from Project Gutenberg
\item[{\href{https://rtweet.info/}{rtweet}}] from Twitter
\item[{\href{https://cran.r-project.org/web/packages/WikipediaR/index.html}{wikipediar}}] from Wikipedia
\end{description}
\end{description}

\noindent\rule{\textwidth}{0.5pt}

Additionally, there are some packages that may not necessarily be useful
for the end user, but may help for our development needs. These
include:
\begin{itemize}
\item \href{https://github.com/bnosac/udpipe}{udpipe} performs
\end{itemize}
tokenisation, parts of speech tagging (which serves as the foundation
for textrank), and more, based on the well-recognised C++
\href{http://ufal.mff.cuni.cz/udpipe}{udpipe library}, using the \href{https://universaldependencies.org}{Universal Treebank}
\begin{itemize}
\item \href{https://github.com/bnosac/BTM}{BTM} performs Biterm Topic Modelling,
\end{itemize}
which is useful for "finding topics in short texts (as occurs in short
survey answers or twitter data)". It uses a somewhat complex sampling
procedure, and like LDA topic modelling, requires a corpus for
comparison. Based on \href{https://github.com/xiaohuiyan/BTM}{C++ BTM} 
\begin{itemize}
\item \href{https://github.com/bnosac/crfsuite}{crfsuite} provides a modelling
\end{itemize}
framework, which is currently outside our current scope, but could be
useful later 
\begin{itemize}
\item In the analysis / removal of names, an important component of a text,
\end{itemize}
\href{https://github.com/ironholds/humaniformat/}{humaniformat} is likely to be useful
\begin{itemize}
\item \href{https://cran.r-project.org/web/views/WebTechnologies.html}{CRAN Task View: Web Technologies and Services} for importing texts from the
\end{itemize}
internet

\subsection{Other Text Analytics Applications}
\label{sec:org7d0c701}
The field of text analytics applications is rather diverse, with most
being general analytics applications with text analytics as a feature of
the application. Some of the applications (general and specific) are
given:

\begin{itemize}
\item \href{http://www.bnosac.be/index.php/products/txtminer}{txtminer} is a
web app for analysing text at a deep level (with something of a
linguistic focus) over multiple languages, for an "educated citizen
researcher"
\end{itemize}

\subsection{Scope Determination}
\label{sec:org9c502de}
The scope of the project is naturally limited by the amount of time
available to do it. As such, exploration of topics such as discourse
analysis, while interesting, is beyond the scope of the project.
Analysis of text must be limited to regular texts, and comparisons
between them. The application must give the greatest amount of insight
to a regular user, in the shortest amount of time, into what the text is
actually about.

\href{http://usresp-student.shinyapps.io/text\_analysis}{Cassidy's project}
was intended to create this, and I have written notes on it
\href{./cassidy\_notes.org}{here}.

Ultimately, I am not completely sold on the idea that term frequencies
and other base-level statistics really give that clear a picture of what
a text is about. It can give some direction, and it can allow for broad
classification of works (eg. a novel will usually have character names
at the highest frequency ranks, scientific works usually have domain
specific terms), but I think word frequencies are less useful to the
analyst than to the algorithms they feed into, such as tf-idf, that may
be more useful. As such, I don't think valuable screen space should be
taken up by low-level statistics such as term frequencies. To me, the
situation is somewhat akin to
\href{https://en.wikipedia.org/wiki/Anscombe's\_quartet}{Anscombe's Quartet}, where the base statistics leave a good deal of information
out, term frequencies being analogous to the modal values.

Additionally, sentiment is really just one part of determining the
semantics of a text. I think too much focus is put on sentiment, which
in practice is something of a "happiness meter". I would like to include
other measurement schemes, such as readability, formality, etc.

Some kind of context in relation to the universal set of texts would be
ideal as well, I think a lot of this analysis occurs in a vacuum, and
insights are hard to come by - something like Google n-grams would be
ideal.

I'm picturing a single page, where the analyst can take one look and
have a fair idea of what a text is about. In reality it will have to be
more complex than that, but that is my lead at the moment. With this in
mind, I want to see keywords, more on \emph{structure} of a text, context,
and clear, punchy graphics showing not \emph{just} sentiment, but several
other key measurements.

\section{Feature Implementations}
\label{sec:org0a78bd3}
\subsection{Introduction}
\label{sec:org4b7876d}
The application essentially consists of a feature-space, with the area
being divided in three; \hyperref[sec:org55af2a7]{Processing}, \hyperref[sec:org24a30ef]{Within-Text Analytics}, and
\hyperref[sec:org986fc5c]{Between-Text Analytics}o. This follows the general format of much of
what is capable in text analysis, and what is of interest to us and our
end users. The UI will likely reflect this, dividing into seperate
windows/panes/tabs to accomodate. Let's look at them in turn:

\subsubsection{Processing}
\label{sec:org55af2a7}
In order for text to be analysed, it must be imported and processed. A
lot of this is an iterative process, coming back for further processing
after analysis etc. Importing will have a "type" selection ability for
the user, where they can choose from a small curated list of easy-access
types, such as gutenberg search, twitter, etc. The option for a custom
text-type is essential, allowing .txt, and for the particularly advanced
end-user, .csv.

Once the file is imported/type is downloaded, the option should exist to
allow the specification of divisions in the text. In a literary work,
these include "chapter", "part", "canto", etc. A twitter type would
allow division by author, by tweet, etc. An important aspect of this
processing is to have a clear picture of what the data should look like.
Division of a text should be associated with some visualisation of the
resulting structure of the text, such as a horizontal bar graph showing
the raw count of text (word count) for each division - this would allow
immediate insight into the correctness of the division, by sighting
obvious errors immediately, and allowing fine tuning so that, for
example, the known number of chapters match up with the number of
divisions. We could implement a few basic division operators in regex,
while following the philosophy of allowing custom input if wanted.
Example regex for "Chapter" could be
\texttt{/[Cc]hapter[.:]?[   ]\{0,10\}-?[  ]\{0,10\}([0-9]|[ivxIVX]*))/g}, something
the end user is likely not wanting to input themselves.

Removal and transformation is another important processing step for
text, with stopwords and lemmatisation being invaluable. The option
should exist to remove specific types of words, which can again come
from prespecified lists. An aspect worth considering is if this should
be done in a table manipulation, or a model - or both, with the length
of the text deciding automatically based on sensible defaults. Again,
the need for a clear picture of the data is essential, with some visual
indication of the data during transformation and removal essential; this
could take the form of some basic statistics, such as a ranking of terms
by frequencies, and some random passage chosen.

Processing multiple documents is also essential. The importation is
something that has to be got right, otherwise it'll be more complex than
it already is, and the end-user will lose interest before the show even
begins. My initial thoughts are of a tabbed import process, with each
tab holding the processing tasks for each individual document, however
this won't scale well to large corpus imports.

\subsubsection{Within-Text Analytics}
\label{sec:org24a30ef}
Within-text analytics should have options to look at the whole text as
it is, whether to look by division, or whether to look at the entire
imported corpus as a whole.

A killer feature here is the production of a summary; a few key
sentences that summarise the text. It's a case of using text to describe
text, but done effectively, it has the potential to compress a large
amount of information into a small, human-understandable object.

Related to the summary, keywords in the text will give a good indication
of topics and tone of the text, as well as perhaps more grammatical
notions, such as authorial word choices. There is the possibility of
using keywords as a basis for other features, such as the ability to use
a search engine to find related texts from the keywords.

Bigrams and associated terms are also excellent indicators of a text.
Something I particularly liked in Cassidy's project was the ability to
search for a term, and see what was related to it. In that case, the
text was "Peter Pan", and searching for a character's name yielded a
wealth of information of the emotions and events attached to the
character.

Sentiment is a feature that has been heavily developed by the field of
text analytics, seeing a broad variety of uses. here, it would be worth
examining sentiment, by word and over the length of the text overall.

\subsubsection{Between-Text Analytics}
\label{sec:org986fc5c}
As in within-text analytics, between-text analytics should have options
for specifying the component of the text that is of interest; here, the
two major categories would be comparisons between divisions within an
individual text, and comparisons between full texts.

Topic modelling gives an idea of what some topics are between texts -
something odd to me is that there isn't a huge amount of information on
topic modelling purely within a text, it always seems to be between
texts (LDA etc.)

tf-idf for a general overview of terms more or less unique to different
texts.

Summarisation between all texts would also be enormously useful.

\subsection{Test Corpus}
\label{sec:org5521762}
It is essential to test on a broad variety of texts in order to create
the most general base application, so a "test set" will have to be
developed. All data is stored in the folder \href{data}{data}

\textbf{Must have}

\begin{itemize}
\item Literature (eg. Dante's Divine Comedy)
\item Survey response data (eg. nzqhs, Cancer Society)
\item Transcript; lack of punctuation may cause difficulties in processing
sentences.
\item Twitter
\end{itemize}

\textbf{Would be nice}

\begin{itemize}
\item article
\begin{itemize}
\item journal (scientific, social)
\item news
\item blog
\item wikipedia
\end{itemize}
\item discourse
\begin{itemize}
\item interview
\item subtitles
\end{itemize}
\item documentation
\begin{itemize}
\item product manual
\item technical user guide
\end{itemize}
\item literature
\begin{itemize}
\item novel
\item play
\item poetry
\end{itemize}
\item textbook
\end{itemize}

\subsection{Text Summarisation}
\label{sec:orgfd5b6b1}

\href{https://en.wikipedia.org/wiki/Automatic\_summarization}{Wikipedia: Automatic Summarisation}

Text summarisation creates enormous insight, especially from a long
text. There are a variety of different techniques, of varying
effectiveness and efficiency. A famous example of automatic text
summarisation comes from \href{https://www.reddit.com/user/autotldr}{autotoldr}, a bot
on reddit that automatically generates summaries of news articles in 4-5
sentences. Autotldr is powered by \href{https://smmry.com/about}{SMMRY},
which explains it's algorithm as working through the following steps:

\begin{enumerate}
\item Associate words with their grammatical counterparts. (e.g "city" and
"cities")
\item Calculate the occurrence of each word in the text.
\item Assign each word with points depending on their popularity.
\item Detect which periods represent the end of a sentence. (e.g "Mr." does
not).
\item Split up the text into individual sentences.
\item Rank sentences by the sum of their words' points.
\item Return X of the most highly ranked sentences in chronological order.
\end{enumerate}

The two main approaches to automatic summarisation are extractive and
abstractive; \textbf{Extractive} uses some subset of the original text to form
a summary, while \textbf{abstractive} techniques form semantic representations
of the text. Here, we will stick to the clearer, simpler, extractive
techniques for now.

\href{https://github.com/bnosac/textrank}{textrank} has the unique idea of
extracting keywords automatically from a text using the pagerank
algorithm (pagerank studied in depth in STATS 320) - my exploration of
the package is documented \href{./textrank\_exploration.Rmd}{here}. At
present, the R implementation of it creates errors for large text files,
but it is worth exploring more into it - whether it is the
implementation, or if it is the algorithm itself.

Hvidfeldt is a prolific blogger focussing on text analysis - he put up
this tutorial on incorporating textrank with tidy methods:
\href{https://www.hvitfeldt.me/blog/tidy-text-summarization-using-textrank/}{tidy textRank}

Further summarisation experimentation is continued
\href{summarisation\_experimentation.Rmd}{here}

\noindent\rule{\textwidth}{0.5pt}

LexRank and textRank appear to exist complimentarily to one another.
Below is a brief summary of how they work

\subsubsection{TextRank}
\label{sec:orgf9ac5fd}

TextRank essentially finds the most representative sentence of a text
based on some similarity measure to other sentence.

By dividing a text into sentences, measures of similarity between every
sentence is calculated (by any number of possible similarity measures),
producing an adjacency matrix of a graph with nodes being sentences,
edge weights being similarity. The PageRank algorithm is then run on
this graph, deriving the best connected sentences, and thereby the most
representative sentences. A list is produced giving sentences with their
corresponding PageRank. The top \(n\) sentences can be chosen, then output
in chronological order, to produce a summary.

In the generation of keywords, the same process described is typically
run on unigrams, with the similarity measure being co-occurance.

\subsubsection{LexRank}
\label{sec:org01a57cc}
LexRank is essentially the same as textRank, however uses
\href{https://en.wikipedia.org/wiki/Cosine\_similarity}{cosine similarity} of tf-idf vectors as it's measure of similarity. LexRank is better at
working across multiple texts, due to the inclusion of a heuristic known
as "Cross-Sentence Information Subsumption (CSIS)"
\end{document}